\documentclass{article}
\usepackage[UTF8,noindent]{ctexcap}

\title{计量经济学成长之路
\\[2ex]
\begin{large}
研究设计学习
\end{large}}

\author{张毅}
\date{\today}

\usepackage{natbib}
\usepackage{graphicx}
\graphicspath{{figures/}}
\usepackage{natbib}
\usepackage{url} %需要和\url{}配合使用
\newpage

\begin{document}

\maketitle

\newpage
\section{父母失业对孩子辍学的影响}
The Effect of Parental Job Loss on Child School Dropout: Evidence from the Occupied Palestinian Territories
\par 摘要:Journal of Development Economics发表的论文利用巴勒斯坦被占地区样本研究父母工作失业对于孩子辍学的影响\citep{di_maio_2019}。
\par 负面的经济冲击如何影响家庭的教育决策?现有研究对发展中国家的关注不够,那不勒斯帕斯诺普大学的Michele Di Maio 和Roberto Nisticò 发表在 Journal of Development Economics 上的论文“The Effect of Parental Job Loss on Child School Dropout: Evidence from the Occupied Palestinian Territories”以巴勒斯坦被占领地区样本,利用巴以冲突导致的巴勒斯坦人失业这一外生冲击克服估计中存在的内生性问题,提供了来自发展中国家的证据。
1993年巴勒斯坦与以色列签订了奥斯陆协议,地区局势实现阶段性和平,巴勒斯坦工人得以进入以色列从事低技能工作,在此期间以色列雇用了超过25%的巴勒斯坦劳动力,占巴勒斯坦国民收入的1/6。而2000年后地区局势恶化,以色列限制巴勒斯坦劳动力进入以色列境内,导致巴勒斯坦被占地区人口失业,作者通过这一外生冲击研究父母失业对于学生辍学的影响 。
\par 研究数据来源于巴勒斯坦劳动力调查数据(PLFS),该数据收集了巴勒斯坦16岁以上劳动力工作相关的季度信息,结合数据中2000-2006年间该数据中10-15岁儿童样本,并利用NGO组织B’Tselem提供的每季度每一万人中由以色列军队造成的死亡人数衡量冲突的严重程度,将其作为巴勒斯坦人口失业的工具变量 。
识别策略:作者实证研究了父母失业对于学生辍学的影响。对于辍学的定义:若孩子t期上学,t+1期未上学则定义为辍学;同样,若受访者t期在以色列境内就业,t+1期在以色列未被雇佣则定义为失业。并控制孩子、家庭特征、地区固定效应以及时间固定效应,进行OLS估计。在此基础上利用冲突强度(死亡率)作为巴勒斯坦被占地区人口失业的工具变量。如果冲突直接影响到辍学率那么工具变量的相关假定将无法满足,比如冲突造成的学校基础设施的破坏以及去学校的困难程度进而造成辍学,如果这一假设成立,那么冲突的影响将会对被占地区所有学生产生影响。,作者比较了父母在以色列工作与父母在被占地区工作的孩子辍学数据发现,冲突只影响父母在以色列工作孩子的辍学率,进而验证工具变量的有效性; 此外,作者发现被占地区的教育系统坚不可摧,冲突地区的学校入学率反而还较前期有所提升,据此得以论证工具变量的有效性。
\par 潜在威胁还在于:第一,学生可能是由于抵抗以色列并非因为父母丢失工作辍学;第二,家庭可能因冲突而决定搬迁,高技能工人的家庭搬到死亡人数较少的地区。对于威胁一,作者通过引用文献佐证入学率与以色列国防军实施的军事措施强度间无相关性,排除了将辍学率与冲突强度联系起来的反馈机制;对于威胁二,作者引用巴勒斯坦统计局数据证明,冲突时期的家庭迁徙不多,且迁徙群体中大部分是出于婚姻而进行的。
\par 在克服内生性问题,并进行一系列稳健型检验(样本扩大、失业定义的变化以及删掉未与以色列接壤的地区样本)后(样本扩大、失业定义的变化以及删掉未与以色列接壤的地区样本)发现的结果表明,父母失业会造成孩子辍学的可能性几率提高9个百分点,异质性方面男生与学业表现较差的学生受父母失业影响较大。

\par 评论:这篇文章的在识别部分写的很清楚,也是我第一篇能看懂的文献


\newpage
\section{饥荒的长期影响}
The Long Term Consequences of Famine on Survivors: Evidence from a Unique Natural Experiment using China's Great Famine
\par 摘要:NBER Working Paper 发表的工作论文采用分位数加工具变量回归,估计出饥荒对幸存者健康的净效应\citep{meng_long_2009}。
\par 1959-1961年间的饥荒造成了大约16-30万中国人死亡,幼儿及母胎期间经历过1959-1961年饥荒的人群现在正处于其中年时期,现有研究侧重于饥荒对死亡率的影响,但对幸存者的研究却比较缺乏(文章于2009年公布最初版本,最近几年研究长期影响的文章多了起来)。世界粮食组织2002年估计非洲约有3800万人口正面临饥荒的威胁,因此研究饥荒的对幸存者的长期影响具有现实意义。澳大利亚国立大学 Meng Xin 与耶鲁大学Nancy Qian 撰写的NBER 工作论文“The Long Term Consequences of Famine on Survivors: Evidence from a Unique Natural Experiment using China's Great Famine”克服了已有研究存在的识别难题后,估计出幼儿及母胎时期经历饥荒对于个体成年后健康和劳动力产出影响的净效应。
\par 研究饥荒对幸存者的长期影响存在以下识别问题,第一,对于饥荒严重程度的测量有偏差;第二,饥荒的影响可能并不随机,饥荒往往发生在粮食产量不高的地区,背后的影响也许是出生地区繁荣与否对于结果变量的影响;第三,饥荒对于幸存者有选择效应,活下来的人群大部分都是比较强壮的群体,而这也与未来的健康产出高度相关。以上三点都会对估计结果有影响。
\par 针对前两个问题,作者首先采用县级层面的人口队列规模来衡量饥荒严重程度,在此基础上使用1997年自然环境和播种面积及不同出生群体上的差异作为饥荒严重程度的工具变量(饥荒往往发生在粮食产量不高的地区);针对问题三,高分位数的估计结果较为准确的反映了饥荒的真实影响,因此采用分位数回归进行分析。
此外,由于人口迁移或者其他外生冲击同时发生从而造成估计结果的不准确。当时我国严格禁止跨地区移民,因此不存在由于人口迁移而影响估计结果的问题;其次,这次饥荒是在一个政治比较稳定的背景下发生的,并无其他类似于战争的影响,因此可以识别出饥荒的净影响。
\par 文章的分析结果表明,母胎或者幼儿时期经历饥荒对成年后的身高、体重、BMI、受教育程度以及劳动力供给有很大的负面影响。



%\newpage
%\section{待定}
%
%\newpage
%\section{待定}
%
%\newpage
%\section{待定}
%
%\newpage
%\section{待定}



\newpage
\section{参考文献}
引用用文献的案例在此处\citet{di_maio_2019}
\\ \citep{di_maio_2019}  注意两类的不同
\bibliographystyle{plainnat}
\bibliography{references}

\end{document}


