%*============================================================*
%**Goal		:   隔代同住对于青少年认知与非认知能力影响研究
%**Author	:  	 ZhangYi zhangyiceee@163.com 15592606739
%**Created	:  	 20200323
%**Last Modified: 2020
%*============================================================*

%!TEX program = xelatex

\documentclass{beamer}
\usepackage[UTF8, noindent, heading]{ctex}
\usepackage{natbib}
\usepackage{hyperref}
\usepackage{multirow} %多行合并为1行


\graphicspath{{figures/}}
%\renewcommand\tablename{表}
\setbeamertemplate{caption}[numbered]

%设置首行缩进
\usepackage{indentfirst}
\setlength{\parindent}{2em} %缩进


\usetheme{Madrid}
%\usecolortheme{crane} %黄色
%\usecolortheme{wolverine} %黄蓝
%\usecolortheme{lily} %蓝白
%Information to be included in the title page:
\title[三代同住、隔代同住]{与祖辈同住对青少年认知与非认知能力影响研究}
\author{张毅}
\date{\today}

\begin{document}
\frame{\titlepage}
%开始你的边际贡献


\begin{frame}
\frametitle{摘要}
文章利用中国教育追踪调查2013-2014年数据研究了家庭居住结构对青少年认知与非认知能力的影响。研究结果表明,相较于核心家庭(父母双方均在家),扩展家庭(父母+祖辈)能显著提高青少年的认知与非认知能力,隔代家庭对于青少年的非认知能力有消极影响,但对其认知能力无显著影响。
\end{frame}


\begin{frame}
	\frametitle{背景}
\begin{itemize}
		\item 青少年发展的重要性,人力资本理论
		\item 家庭在青少年发展中具有至关重要的作用
\end{itemize}
\par 1966年《科尔曼报告》发布后,教育领域的学者们将研究从学校扩展至家庭,在家庭领域的相关研究中,学者们都认为在控制了父辈的因素后,祖辈不会对孙辈产生影响,这也就是典型的马尔可夫过程。
\par 随着社会经济的发展,我国家庭居住结构也发生了变化,三代家庭以及隔代家庭的增加,学者们也呼吁应关注祖辈对于孙辈的影响\citep{mare_multigenerational_2011,张帆2020,zeng_effects_2014}
\end{frame}
%其他办法
\begin{frame}
	\frametitle{背景}

	\begin{itemize}
	\item 在计划经济时代,单位提供个体从坟墓的福利,支持劳动者的家务、照料等社会再生产过程,极大缓解家庭的困难和女性的工作--家庭冲突\citep{ji_young_2020}。
	\item 随着我国从计划经济转向市场经济,原来由单位承担的社会福利体系逐渐市场化和私人化。
	\item 市场化过程中,我国目前并未建立起完善的社会保障制度和家务劳动市场\citep{张帆2020},因此对于大多数家庭而言,寻求祖辈照顾孙辈成为理性选择。 
\end{itemize}

\end{frame}



\begin{frame}
	\frametitle{现有研究观点} %感觉这块用处不大
Children raised in two-biological-parent families tend to fare better academically than children raised in any other family form\citep{brown_marriage_2010}许多研究都表明在双亲家庭在学业上都比其他类型的居住模式好
\par 家庭结构在是那个方面有影响(可以作为中间变量,来做机制检验)
\begin{itemize}
	\item 家庭经济条件:单亲家庭经济处于弱势
	\item 父母社会化:家长社会化解释的重点是育儿,首先是父母的教养实践。这里的论点是,与双亲家庭相比,单亲家庭、继父家庭和同居家庭中通常存在较弱的父母权威结构,监控和监督较少,这反过来又降低了儿童接受的养育质量
	\item 家庭不稳定:家庭结构的变化会造成负面影响,与家庭类型无关
\end{itemize}
\end{frame}

\begin{frame}
\frametitle{Problem} 
%这一部分基本上是文献的梳理部分
	\begin{itemize}
		\item 与祖辈同住会不会对青少年产生影响?
		\item 国外相关文献较多
		\begin{itemize}
			\item 但国外居住成因与我国有差异
			\item 父母吸毒、犯罪等造成的与祖辈同住
 		\end{itemize}
		\item 国内相关研究刚起步,具有代表性的是吴愈晓\citep{吴愈晓2018,张帆2020}、谢宇\citep{zeng_effects_2014}以及来自台湾样本的研究\citep{zeng_effects_2014}
		\begin{itemize}
			\item 样本中存在的问题是低龄化,对于青少年相关研究关注不够
			\item 尽管国内学者也进行了探索,存在的问题是目前在定义上存在差异,如\citet{张帆2020}的定义中混有了隔代居住的样本,但本文认为隔代同住与三代同住之间仍存在结构性差异。
			\item 关注健康和认知能力较多,但是对于个体的非认知能力涉及不多(Heckman GED考试),认知与非认知能力对个体的未来收入有很好的预测作用,因此研究非认知能力非常重要
		\end{itemize}	
	\end{itemize}
\end{frame}

\begin{frame}
	\frametitle{Objective}
	本文的目标是研究与祖辈同住对于青少年发展的影响
	\begin{itemize}
		\item 描述三代同住、隔代同住和亲代同住之间在特征方面的差异;
		\item 研究与祖辈同住对青少年认知能力的影响;
		\item 研究与祖辈同住对青少年非认知能力的影响
	\end{itemize}
\end{frame}


\begin{frame}
	\frametitle{数据及主要变量定义}
数据来自中国教育追踪调查基线(2013-2014年)数据,运用多元线性回归并辅之以学校固定效应模型。
根据学生问卷中“这学期和你在家住的都有:”进行判断
	\begin{itemize}
		\item 亲代同住:与父亲和母亲同住\citep{brown_marriage_2010}
		\item 三代同住(拓展家庭):父亲+母亲+祖辈
		\item 隔代同住:只与祖辈同住
		\item 其他(不在本文研究范围内)
	\end{itemize}
	认知能力测量根据CEPS的调查直接使用;非认知能力借鉴前人研究进行借鉴,
\end{frame}

\begin{frame}
	\frametitle{变量定义:认知能力}
CEPS的认知测试主要测量学生的逻辑思维与问题解决能力,不涉及学校课程的相关测试,各校间可直接比较,因此在本研究中直接使用认知测试得分来衡量青少年的认知能力。
\end{frame}


\begin{frame}
	\frametitle{变量定义:非认知能力}
\begin{table}[!htbp]
\caption{非认知能力的度量}
\label{nocog_scale}
\resizebox{190pt}{40pt}{
\begin{tabular}{c|l}
\hline
\multirow{5}{*}{情绪稳定性} & 沮丧                             \\
                       & 抑郁                             \\
                       & 不快乐                            \\
                       & 生活没有意思                         \\
                       & 悲伤                             \\ \hline
\multirow{3}{*}{尽责性}   & 就算身体有点不舒服或者有其他理由可以留在家中仍然会尽量去上学 \\
                       & 就算不喜欢的功课用也会尽全力去学               \\
                       & 就算功课需要花好长时间才能做完,仍然会不断的尽力去做     \\ \hline
\multirow{4}{*}{宜人性}   & 班里大多数同学对我很友好                   \\
                       & 我所在班级班风良好                      \\
                       & 我经常参加学校或班级组织的活动                \\
                       & 我对这个学校的人感到亲近                   \\ \hline
\end{tabular}}
\end{table}
非认知能力的衡量,本文采用现有文献的通行做法,对以上题目进行主成分因子分析后发现可以聚合为一个公因子,相应得分代表该纬度大五人格特征越强。
\end{frame}


\begin{frame}
	\frametitle{实证模型}
\begin{equation}
	Y_{ij}=\beta_1Three_{ij}+\beta_2Inter_{ij}+Child_{ij}+Parent_{ij}+School_j+\varepsilon_{ij}
\end{equation}

$Y_{ij}$ 代表第$j$个学校第i个学生认知或非认知能力;$Three_ij$表示第$j$个学校第$i$个学生是否生活在三代同住家庭(“三代同住”=1,“非三代同住”=0);$Inter_{ij}$表示第$j$个学校第$i$个学生是否只与祖辈同住(“隔代同住”=1,“非隔代同住”=0);$Child_{ij}$为学生个体层面的控制变量,包括性别、年龄、民族、是否为独生子女、是否为9年级、户口状况、是否上过学前班以及是否为流动儿童;$Parent_{ij}$为第$j$个学校第$i$个学生父母的最高受教育程度以及家庭经济条件的虚拟变量,$School_i$为学校层面的固定效应。
\end{frame}


\begin{frame}
\centering{\huge{结果}}
\end{frame}

\begin{frame}
\begin{table}[!htbp]
\centering
\caption{与祖辈同住对青少年认知能力的影响}
\label{cog1}
\resizebox{120pt}{90pt}{
\begin{tabular}{lccc}
\hline
		    & \multicolumn{3}{c}{认知能力}  \\

             & (1)       & (2)        & (3)       \\
\hline
             &           &            &           \\
三代同住         & 0.0627*** & 0.0584***  & 0.0503*** \\
             & (0.0163)  & (0.0161)   & (0.0154)  \\
完全隔代(只和祖辈同住) & -0.0629** & -0.0370    & 0.0203    \\
             & (0.0257)  & (0.0254)   & (0.0251)  \\
年龄           & -0.189*** & -0.163***  & -0.118*** \\
             & (0.0100)  & (0.0100)   & (0.0118)  \\
少数民族         & -0.109*** & -0.0805*** & -0.0558   \\
             & (0.0257)  & (0.0255)   & (0.0344)  \\
女性           & 0.00951   & 0.00333    & -0.0146   \\
             & (0.0138)  & (0.0136)   & (0.0136)  \\
初三           & 0.368***  & 0.321***   & 0.255***  \\
             & (0.0244)  & (0.0243)   & (0.0398)  \\
农村           & -0.135*** & -0.0409*** & 0.0216    \\
             & (0.0148)  & (0.0155)   & (0.0189)  \\
上过学前班        & 0.171***  & 0.142***   & 0.108***  \\
             & (0.0179)  & (0.0178)   & (0.0180)  \\
流动儿童         & 0.00648   & 0.00690    & -0.0163   \\
             & (0.0179)  & (0.0178)   & (0.0249)  \\
独生子女         & 0.234***  & 0.162***   & -0.00455  \\
             & (0.0151)  & (0.0155)   & (0.0197)  \\
父母的最高受教育年限   &           & 0.0408***  & 0.0248*** \\
             &           & (0.00264)  & (0.00357) \\
经济条件中等       &           & 0.130***   & 0.0371**  \\
             &           & (0.0180)   & (0.0178)  \\
经济条件富裕       &           & 0.181***   & 0.00849   \\
             &           & (0.0314)   & (0.0328)  \\
Constant     & 2.440***  & 1.542***   & 1.258***  \\
             & (0.140)   & (0.147)    & (0.181)   \\
             &           &            &          \\
学校固定效应  & 否    	& 否     	& 是  \\
R-squared    & 0.094     & 0.115      & 0.243 \\
Observations & 14,208    & 14,208     & 14,208    \\
\hline
\end{tabular}}
\end{table}
\end{frame}




\begin{frame}
\begin{table}[htbp]
\centering
\caption{与祖辈同住对青少年非认知能力的影响}
\label{nocog}
\resizebox{\textwidth}{!}{
\begin{tabular}{lccccccccc}
\hline
	    & \multicolumn{3}{c}{情绪稳定性}    & \multicolumn{3}{c}{宜人性}  & \multicolumn{3}{c}{尽责性}      \\
             & (1)        & (2)        & (3)        & (4)        & (5)        & (6)       & (7)        & (8)        & (9)       \\
\hline
三代同住         & 0.0454**   & 0.0421**   & 0.0513**   & 0.0602***  & 0.0557***  & 0.0603*** & 0.0489**   & 0.0497**   & 0.0407*   \\
             & (0.0199)   & (0.0198)   & (0.0226)   & (0.0195)   & (0.0194)   & (0.0188)  & (0.0195)   & (0.0195)   & (0.0222)  \\
完全隔代(只和祖辈同住) & -0.175***  & -0.164***  & -0.127***  & -0.121***  & -0.0970*** & -0.0457   & 0.0107     & 0.00696    & -0.00163  \\
             & (0.0312)   & (0.0312)   & (0.0345)   & (0.0307)   & (0.0305)   & (0.0329)  & (0.0307)   & (0.0308)   & (0.0333)  \\
年龄           & -0.0678*** & -0.0584*** & -0.0412*** & -0.0614*** & -0.0388*** & -0.0136   & -0.0180    & -0.0216*   & -0.0128   \\
             & (0.0122)   & (0.0123)   & (0.0128)   & (0.0120)   & (0.0120)   & (0.0143)  & (0.0120)   & (0.0122)   & (0.0112)  \\
少数民族         & -0.0573*   & -0.0366    & 0.0453     & -0.185***  & -0.153***  & -0.0389   & -0.0106    & -0.0157    & -0.00831  \\
             & (0.0313)   & (0.0313)   & (0.0346)   & (0.0308)   & (0.0306)   & (0.0408)  & (0.0308)   & (0.0309)   & (0.0473)  \\
女性           & -0.0684*** & -0.0723*** & -0.0774*** & 0.155***   & 0.150***   & 0.144***  & 0.241***   & 0.242***   & 0.239***  \\
             & (0.0167)   & (0.0167)   & (0.0180)   & (0.0164)   & (0.0163)   & (0.0155)  & (0.0165)   & (0.0165)   & (0.0178)  \\
初三           & -0.0865*** & -0.107***  & -0.138***  & 0.0751**   & 0.0316     & -0.00432  & -0.285***  & -0.279***  & -0.289*** \\
             & (0.0297)   & (0.0298)   & (0.0360)   & (0.0292)   & (0.0292)   & (0.0393)  & (0.0293)   & (0.0294)   & (0.0303)  \\
农村           & 0.0415**   & 0.0684***  & 0.0484**   & -0.0688*** & 0.00880    & -0.00700  & 0.0885***  & 0.0765***  & 0.0618*** \\
             & (0.0180)   & (0.0190)   & (0.0214)   & (0.0176)   & (0.0186)   & (0.0203)  & (0.0177)   & (0.0188)   & (0.0190)  \\
上过学前班        & 0.0955***  & 0.0797***  & 0.0671***  & 0.148***   & 0.118***   & 0.103***  & -0.0177    & -0.0131    & -0.00586  \\
             & (0.0218)   & (0.0218)   & (0.0240)   & (0.0214)   & (0.0213)   & (0.0208)  & (0.0214)   & (0.0215)   & (0.0219)  \\
流动儿童         & -0.0153    & -0.0239    & -0.0234    & 0.0204     & 0.0148     & 0.0287    & -0.0812*** & -0.0803*** & -0.0138   \\
             & (0.0218)   & (0.0218)   & (0.0233)   & (0.0214)   & (0.0213)   & (0.0240)  & (0.0215)   & (0.0215)   & (0.0221)  \\
独生子女         & 0.0436**   & 0.0187     & 0.00372    & 0.139***   & 0.0790***  & -0.00980  & 0.00581    & 0.0154     & -0.0156   \\
             & (0.0184)   & (0.0190)   & (0.0224)   & (0.0181)   & (0.0186)   & (0.0206)  & (0.0181)   & (0.0188)   & (0.0219)  \\
父母的最高受教育年限   &            & 0.00445    & 0.00821**  &            & 0.0280***  & 0.0191*** &            & -0.00435   & -0.00324  \\
             &            & (0.00325)  & (0.00392)  &            & (0.00317)  & (0.00374) &            & (0.00320)  & (0.00387) \\
经济条件中等       &            & 0.189***   & 0.160***   &            & 0.194***   & 0.130***  &            & -0.0330    & -0.0564** \\
             &            & (0.0222)   & (0.0263)   &            & (0.0217)   & (0.0206)  &            & (0.0219)   & (0.0231)  \\
经济条件富裕       &            & 0.220***   & 0.159***   &            & 0.350***   & 0.217***  &            & -0.0470    & -0.0843** \\
             &            & (0.0386)   & (0.0432)   &            & (0.0377)   & (0.0384)  &            & (0.0380)   & (0.0387)  \\
Constant     & 0.957***   & 0.637***   & 0.405**    & 0.659***   & -0.108     & -0.256    & 0.266      & 0.386**    & 0.279     \\
             & (0.171)    & (0.181)    & (0.191)    & (0.168)    & (0.177)    & (0.207)   & (0.168)    & (0.178)    & (0.174)   \\
学校固定效应  & 否    	& 否     	& 是 			& 否    	 	& 否     	 & 是    		& 否    		& 否     	& 是  \\
R-squared    & 0.023      & 0.029      & 0.061      & 0.033      & 0.048      & 0.109     & 0.043      & 0.043      & 0.072    \\
Observations & 14,208     & 14,208     & 14,208     & 14,208     & 14,208     & 14,208    & 14,208     & 14,208     & 14,208    \\
\hline
\end{tabular}} %注意这里还有}
\end{table}	
\end{frame}



\begin{frame}
	\frametitle{Discussion}
	可以看出三代同住对于青少年认知能力有显著的促进作用,隔代同住对于青少年认知能力没有显著影响
\\ 但从非认知能力来看,三代同住的家庭仍旧显示出对于青少年的正向影响,但是隔代同住对于青少年的情绪稳定性和宜人性有显著的负面影响。
\end{frame}

\begin{frame}
	\frametitle{局限}
	\begin{itemize}
		\item 数据的限制无法控制祖辈的相关信息
		\item 机制检验文中没有涉及,准备接下来进行补充

	\end{itemize}
\end{frame}


\begin{frame}
    \frametitle{参考文献}
\tiny
\bibliographystyle{plainnat}
\bibliography{references}
\end{frame}
%参考文献的案例 \citet{Krueger1999Experimental} \citep{Krueger1999Experimental}  注意两类的不同
\end{document}



