%*============================================================*
%**Goal		:   隔代同住对于青少年认知与非认知能力影响研究
%**Author	:  	 ZhangYi zhangyiceee@163.com 15592606739
%**Created	:  	 20200323
%**Last Modified: 2020
%*============================================================*

%!TEX program = xelatex

\documentclass{beamer}
\usepackage[UTF8, noindent, heading]{ctex}
\usepackage{natbib}
\usepackage{hyperref}
\usepackage{multirow} %多行合并为1行


\graphicspath{{figures/}}
%\renewcommand\tablename{表}
\setbeamertemplate{caption}[numbered]

%设置首行缩进
\usepackage{indentfirst}
\setlength{\parindent}{2em} %缩进


\usetheme{Madrid}
%\usecolortheme{crane} %黄色
%\usecolortheme{wolverine} %黄蓝
%\usecolortheme{lily} %蓝白
%\usecolortheme{spruce} %绿色
%\usecolortheme{beaver} %红色
%Information to be included in the title page:
\title[与祖辈同住]{与祖辈同住对青少年认知与非认知能力影响研究}
\author[CEEE沙龙]{}

\date{\today}

\begin{document}
\frame{\titlepage}
%开始你的边际贡献


\begin{frame}
\frametitle{摘要}
探究影响青少年发展的因素具有现实意义,文章从家庭结构的视角出发,利用中国教育追踪调查数据,运用多元线性回归模型分析了不同家庭结构对于学生认知与非认知能力的影响。结果表明,在控制了学生个体、家庭特征及学校固定效应后,三代同住家庭对于青少年认知与非认知均有显著的促进作用,隔代同住家庭对于青少年非认知能力有消极影响,但对其认知能力无显著影响。
\end{frame}

\begin{frame}
	\frametitle{背景}
	改革开放以来,我国三代家庭、隔代家庭的比例呈上升趋势\citep{wang_2013}。
		\begin{itemize}
		\item 在计划经济时代,单位提供个体从出生到坟墓的福利,支持劳动者的家务、照料等社会再生产过程,极大缓解家庭的困难和女性的工作--家庭冲突\citep{ji_young_2020}。
		\item 随着我国从计划经济转向市场经济,原来由单位承担的社会福利体系逐渐市场化和私人化。
		\item 而在市场化过程中,我国并未建立起完善的社会保障制度和家务劳动市场\citep{张帆2020},因此对于大多数家庭而言,寻求祖辈照顾孙辈成为理性选择。 
	\end{itemize}
\end{frame}



\begin{frame}
	\frametitle{背景}
\par 1966年《科尔曼报告》发布后,教育领域的学者们将关注点从学校资源转向家庭方面。在家庭领域的相关研究中,很少有学者关注到祖辈对于青少年发展的影响,在控制了父辈的因素后,祖辈不会对孙辈产生影响(马尔可夫过程)。近年来,学者们从马尔可夫过程转向了非马尔可夫过程,来研究祖辈对孙辈的直接影响\citep{mare_multigenerational_2011}。

\end{frame}
%其他办法



%\begin{frame}
%	\frametitle{现有研究观点} %感觉这块用处不大,或者位置不对。
%Children raised in two-biological-parent families tend to fare better %academically than children raised in any other family form\citep{brown_marriage_2010}许多研究都表明在双亲家庭在学业上都比其他类型的居住模式好
%\par 家庭结构在是那个方面有影响(可以作为中间变量,来做机制检验)
%\begin{itemize}
%	\item 家庭经济条件:单亲家庭经济处于弱势
%	\item 父母社会化:家长社会化解释的重点是育儿,首先是父母的教养实践。这里的论点是,与双亲%家庭相比,单亲家庭、继父家庭和同居家庭中通常存在较弱的父母权威结构,监控和监督较少,这反过%来又降低了儿童接受的养育质量
%	\item 家庭不稳定:家庭结构的变化会造成负面影响,与家庭类型无关
%\end{itemize}
%\end{frame}

\begin{frame}
\frametitle{文献} 
%这一部分基本上是文献的梳理部分
国外学者做了大量探讨\citep{hayslip_grandparents_2005,monserud_household_2011,westphal_what_2015,jappens_parental_2016,daniela_del_boca_role_2017,zamberletti_grandparents_2018}。
\par 但国外居住成因与我国有差异,如父母吸毒、犯罪等造成的与祖辈同住,且欧美的照料和社会再生产责任很大程度上交给了市场解决。
\par 中国特色的传统与现代的家庭代际关系\citep{shi_2016,ji_2019}:
	\begin{itemize}
		\item 代际交换:年老的父母为已婚子女照料孩子, 而子女为父母提供住处和食物等\citep{wang_2008};
		\item 文化因素:\citep{chu_coresidence_2011}含饴弄孙,中国的祖辈参与照料祖辈的比率远高于其他亚洲国家\citep{zhou_2016};
		\item 女性劳动参与率提高\citep{lv_2016},家务劳动市场不完善寻求祖辈支持;
		\item 留守儿童:大量劳动力由农村流向城市,家中儿童留给祖辈抚养。
	\end{itemize}
\end{frame}
\begin{frame}
\frametitle{文献} 
	国内有关居住安排对老年人影响的研究较多,但是对于青少年影响的研究不多\citep{pong_co-resident_2010,zeng_effects_2014,liang_2017,he_childhood_2018,张帆2020}。
	\par \citet{pong_co-resident_2010}使用我国台湾省样本,发现隔代同住对青少年学业表现有显著正向影响,\citet{zeng_effects_2014}利用CHIP2002年农村样本,指出祖辈对孙辈的影响只在同住的样本中显著,在此基础上\citet{张帆2020}发现三代同住对青少年学业表现有正向影响。
		
\end{frame}

\begin{frame}
\frametitle{文献} 
	\begin{itemize}
			\item 样本代表性不足\citep{pong_co-resident_2010},这点在后来文献中已经解决;
			\item 定义模糊,\citet{张帆2020}三代同住定义中混有了隔代居住的样本,但本文认为隔代同住与三代同住之间仍存在结构性差异,不能一概而论;
			\item 被解释变量多为健康\citep{liang_2017,he_childhood_2018}和学业表现\citep{张帆2020},对于个体的非认知能力涉及不够;
			\item 非认知能力在个体未来发展中发挥着重要作用\citep{heckman_effects_2006,kautz_fostering_2014,赖德胜_2020},在研究中同时考虑认知与非认知能力已被很多学者采用\citep{huang_2017,le_2017}。
	\end{itemize}

\end{frame}

\begin{frame}
	\frametitle{Problem}
	在同时考虑三代同住与隔代同住的情况下,家庭居住安排如何影响青少年的发展?
\end{frame}

\begin{frame}
	\frametitle{Objective}
本文的目标是使用全国代表性样本研究与祖辈同住对于青少年发展的影响
	\begin{itemize}
		\item 研究与祖辈同住对青少年认知能力的影响;
		\item 研究与祖辈同住对青少年非认知能力的影响
	\end{itemize}
\end{frame}


\begin{frame}
	\frametitle{数据及主要变量定义}
数据来自中国教育追踪调查(CEPS)基线(2013-2014年)数据,该数据以7年级和9年级在校学生为调查对象,在全国随机抽取28个县级单位,共112所学校438个班级的约2万名学生,在排除其他居住结构的样本以及删除掉不完整的样本后,进入本研究样本共14208个。
\par 根据学生问卷中“这学期和你在家住的都有:”(多选题)进行判断
	\begin{itemize}
		\item 亲代同住:与父亲和母亲同住\citep{brown_marriage_2010,liang_2017}
		\item 三代同住:父亲+母亲+祖辈
		\item 隔代同住:只与祖辈同住
		\item 其他(不在本文研究范围内)
	\end{itemize}
\par 控制变量:性别、年龄、民族、独生子女、年级、户口、是否上过学前班、是否为流动儿童、父母的最高受教育程度、家庭经济条件的虚拟变量
\end{frame}



\begin{frame}
	\frametitle{变量定义:认知能力}
CEPS的认知测试主要测量学生的逻辑思维与问题解决能力,不涉及学校课程的相关测试,各校间可直接比较,因此在本研究中直接使用认知测试得分来衡量青少年的认知能力。
\end{frame}


\begin{frame}
	\frametitle{变量定义:非认知能力}
\begin{table}[!htbp]
\caption{非认知能力的度量}
\label{nocog_scale}
\resizebox{200pt}{40pt}{
\begin{tabular}{c|l}
\hline
\multirow{5}{*}{情绪稳定性} & 沮丧                             \\
                       & 抑郁                             \\
                       & 不快乐                            \\
                       & 生活没有意思                         \\
                       & 悲伤                             \\ \hline
\multirow{3}{*}{尽责性}   & 就算身体有点不舒服或者有其他理由可以留在家中仍然会尽量去上学 \\
                       & 就算不喜欢的功课用也会尽全力去学               \\
                       & 就算功课需要花好长时间才能做完,仍然会不断的尽力去做     \\ \hline
\multirow{4}{*}{宜人性}   & 班里大多数同学对我很友好                   \\
                       & 我所在班级班风良好                      \\
                       & 我经常参加学校或班级组织的活动                \\
                       & 我对这个学校的人感到亲近                   \\ \hline
\end{tabular}}
\end{table}
\par CEPS数据中并没有专门的量表来测试非认知能力,因此本文采用现有文献的通行做法\citep{huang_2017,le_2017,li_2015,min_2019,li_2020},对表\ref{nocog_scale}中题目进行主成分因子分析后聚合为一个公因子,相应得分代表该纬度的非认知能力,具体见表\ref{nocog_scale}。
\end{frame}



\begin{frame}
	\frametitle{实证模型}
	本文尚未找到导致家庭居住结构发生变化且可以获取相关数据的外生冲击变量,现有的研究条件只提供了有限的观测性数据,但可以通过控制影响家庭居住安排的因素来消除自选择和遗漏变量带来的估计误差。本研究使用如下模型进行估计:
\begin{equation}
	Y_{ij}=\beta_1Three_{ij}+\beta_2Inter_{ij}+Child_{ij}+Parent_{ij}+School_j+\varepsilon_{ij}
\end{equation}
$Y_{ij}$ 代表第$j$个学校第i个学生认知或非认知能力;$Three_{ij}$表示第$j$个学校第$i$个学生是否生活在三代同住家庭(三代同住=1,非三代同住=0);$Inter_{ij}$表示第$j$个学校第$i$个学生是否只与祖辈同住(隔代同住=1,非隔代同住=0);$Child_{ij}$为学生个体层面的控制变量,包括性别(女=1,男=0)、年龄、民族(少数民族=1,汉族=0)、是否为独生子女(独生子女=1,非独生子女=0)、是否为初三(初三=1,初一=0)、户口状况(农村户口=1,非农户口=0)、是否上过学前班(是=1,否=0)以及是否为流动儿童(是=1,否=0);$Parent_{ij}$为第$j$个学校第$i$个学生父母的最高受教育程度以及家庭经济条件的虚拟变量,为排除未被观测到的学校特征对学业表现的影响,加入学校层面的固定效应$School_j$进行估计。 
\end{frame}




\begin{frame}
\begin{table}[!htbp]
\caption{所有变量的描述性统计结果}
\label{sum}
\resizebox{160pt}{100pt}{
\begin{tabular}{lccccc}
\hline
 变量名        &  观测值   &  均值      &  标准差   &  最小值    &  最大值   \\
\hline
亲代同住       &  14208 &  0.679   &  0.467 &  0      &  1     \\ 三代同住       &  14208 &  0.24    &  0.427 &  0      &  1     \\ 隔代同住       &  14208 &  0.0812  &  0.273 &  0      &  1     \\
 年龄         &  14208 &  14.51   &  1.237 &  12     &  18    \\
 女性         &  14208 &  0.503   &  0.5   &  0      &  1     \\
 少数民族       &  14208 &  0.0794  &  0.27  &  0      &  1     \\
 独生子女       &  14208 &  0.45    &  0.498 &  0      &  1     \\
 九年级        &  14208 &  0.487   &  0.5   &  0      &  1     \\
 农村户口       &  14208 &  0.5     &  0.5   &  0      &  1     \\
 上过学前班      &  14208 &  0.815   &  0.389 &  0      &  1     \\
 流动儿童       &  14208 &  0.184   &  0.387 &  0      &  1     \\
 住校生        &  14208 &  0.305   &  0.46  &  0      &  1     \\
 父母的最高受教育年限 &  14208 &  10.91   &  3.047 &  0      &  19    \\
 家庭经济条件困难   &  14208 &  0.189   &  0.391 &  0      &  1     \\
 家庭经济条件中等   &  14208 &  0.746   &  0.435 &  0      &  1     \\
 家庭经济条件富裕   &  14208 &  0.0656  &  0.248 &  0      &  1     \\
 认知能力得分     &  14208 &  0.0649  &  0.851 &  -2.029 &  2.71  \\
 情绪稳定性      &  14208 &  0.00327 &  0.997 &  -3.562 &  1.315 \\
 宜人性        &  14208 &  0.0251  &  0.984 &  -3.121 &  1.36  \\
 尽责性        &  14208 &  0.0156  &  0.991 &  -3.461 &  1.017 \\
 \hline
\end{tabular}}
\end{table}
\end{frame}


\begin{frame} %还在考虑需要不需要这个t检验
\begin{table}
\centering
\caption{均值检验}
\label{ttest}
\resizebox{270pt}{110pt}{
	\begin{tabular}{@{\extracolsep{5pt}}lccccccccc}
\\[-1.8ex]\hline  \\[-1.8ex]
  & \multicolumn{2}{c}{(1)}  & \multicolumn{2}{c}{(2)}  & \multicolumn{2}{c}{(3)}  & \multicolumn{3}{c}{T-test}  \\
 & \multicolumn{2}{c}{亲代同住}  & \multicolumn{2}{c}{三代同住}  & \multicolumn{2}{c}{隔代同住}  & \multicolumn{3}{c}{Difference}  \\
    Variable & N & Mean/SE  & N & Mean/SE  & N & Mean/SE   & (1)-(2)  & (1)-(3)  & (2)-(3)  \\ \hline \\[-1.8ex] 
认知能力 & 9651 & \begin{tabular}[t]{@{}c@{}} 0.058 \\ (0.009) \end{tabular} & 3403 & \begin{tabular}[t]{@{}c@{}} 0.150 \\ (0.014) \end{tabular} & 1154 & \begin{tabular}[t]{@{}c@{}} -0.127 \\ (0.024) \end{tabular} &    -0.092*** &     0.184*** &     0.276*** \rule{0pt}{0pt}\\
情绪稳定性 & 9651 & \begin{tabular}[t]{@{}c@{}} 0.003 \\ (0.010) \end{tabular} & 3403 & \begin{tabular}[t]{@{}c@{}} 0.068 \\ (0.016) \end{tabular} & 1154 & \begin{tabular}[t]{@{}c@{}} -0.183 \\ (0.029) \end{tabular} &    -0.065*** &     0.186*** &     0.251*** \rule{0pt}{3ex}\\
宜人性 & 9651 & \begin{tabular}[t]{@{}c@{}} 0.021 \\ (0.010) \end{tabular} & 3403 & \begin{tabular}[t]{@{}c@{}} 0.103 \\ (0.017) \end{tabular} & 1154 & \begin{tabular}[t]{@{}c@{}} -0.171 \\ (0.029) \end{tabular} &    -0.082*** &     0.192*** &     0.274*** \rule{0pt}{3ex}\\
尽责性 & 9651 & \begin{tabular}[t]{@{}c@{}} -0.007 \\ (0.010) \end{tabular} & 3403 & \begin{tabular}[t]{@{}c@{}} 0.074 \\ (0.017) \end{tabular} & 1154 & \begin{tabular}[t]{@{}c@{}} 0.036 \\ (0.027) \end{tabular} &    -0.082*** &    -0.043 &     0.039 \rule{0pt}{3ex}\\
年龄 & 9651 & \begin{tabular}[t]{@{}c@{}} 14.535 \\ (0.013) \end{tabular} & 3403 & \begin{tabular}[t]{@{}c@{}} 14.382 \\ (0.021) \end{tabular} & 1154 & \begin{tabular}[t]{@{}c@{}} 14.640 \\ (0.039) \end{tabular} &     0.153*** &    -0.104*** &    -0.257*** \rule{0pt}{3ex}\\
少数民族 & 9651 & \begin{tabular}[t]{@{}c@{}} 0.080 \\ (0.003) \end{tabular} & 3403 & \begin{tabular}[t]{@{}c@{}} 0.070 \\ (0.004) \end{tabular} & 1154 & \begin{tabular}[t]{@{}c@{}} 0.106 \\ (0.009) \end{tabular} &     0.010* &    -0.026*** &    -0.036*** \rule{0pt}{3ex}\\
女性 & 9651 & \begin{tabular}[t]{@{}c@{}} 0.496 \\ (0.005) \end{tabular} & 3403 & \begin{tabular}[t]{@{}c@{}} 0.530 \\ (0.009) \end{tabular} & 1154 & \begin{tabular}[t]{@{}c@{}} 0.486 \\ (0.015) \end{tabular} &    -0.034*** &     0.010 &     0.043** \rule{0pt}{3ex}\\
九年级 & 9651 & \begin{tabular}[t]{@{}c@{}} 0.500 \\ (0.005) \end{tabular} & 3403 & \begin{tabular}[t]{@{}c@{}} 0.458 \\ (0.009) \end{tabular} & 1154 & \begin{tabular}[t]{@{}c@{}} 0.475 \\ (0.015) \end{tabular} &     0.042*** &     0.025 &    -0.017 \rule{0pt}{3ex}\\
农村户口 & 9651 & \begin{tabular}[t]{@{}c@{}} 0.478 \\ (0.005) \end{tabular} & 3403 & \begin{tabular}[t]{@{}c@{}} 0.500 \\ (0.009) \end{tabular} & 1154 & \begin{tabular}[t]{@{}c@{}} 0.682 \\ (0.014) \end{tabular} &    -0.022** &    -0.204*** &    -0.182*** \rule{0pt}{3ex}\\
上过学前班 & 9651 & \begin{tabular}[t]{@{}c@{}} 0.814 \\ (0.004) \end{tabular} & 3403 & \begin{tabular}[t]{@{}c@{}} 0.839 \\ (0.006) \end{tabular} & 1154 & \begin{tabular}[t]{@{}c@{}} 0.744 \\ (0.013) \end{tabular} &    -0.025*** &     0.071*** &     0.096*** \rule{0pt}{3ex}\\
流动儿童 & 9651 & \begin{tabular}[t]{@{}c@{}} 0.220 \\ (0.004) \end{tabular} & 3403 & \begin{tabular}[t]{@{}c@{}} 0.118 \\ (0.006) \end{tabular} & 1154 & \begin{tabular}[t]{@{}c@{}} 0.075 \\ (0.008) \end{tabular} &     0.102*** &     0.145*** &     0.043*** \rule{0pt}{3ex}\\
独生子女 & 9651 & \begin{tabular}[t]{@{}c@{}} 0.453 \\ (0.005) \end{tabular} & 3403 & \begin{tabular}[t]{@{}c@{}} 0.511 \\ (0.009) \end{tabular} & 1154 & \begin{tabular}[t]{@{}c@{}} 0.244 \\ (0.013) \end{tabular} &    -0.058*** &     0.210*** &     0.268*** \rule{0pt}{3ex}\\
父母最高受教育年限 & 9651 & \begin{tabular}[t]{@{}c@{}} 10.977 \\ (0.032) \end{tabular} & 3403 & \begin{tabular}[t]{@{}c@{}} 11.150 \\ (0.051) \end{tabular} & 1154 & \begin{tabular}[t]{@{}c@{}} 9.693 \\ (0.074) \end{tabular} &    -0.174*** &     1.284*** &     1.457*** \rule{0pt}{3ex}\\
经济条件中等 & 9651 & \begin{tabular}[t]{@{}c@{}} 0.747 \\ (0.004) \end{tabular} & 3403 & \begin{tabular}[t]{@{}c@{}} 0.771 \\ (0.007) \end{tabular} & 1154 & \begin{tabular}[t]{@{}c@{}} 0.661 \\ (0.014) \end{tabular} &    -0.024*** &     0.086*** &     0.110*** \rule{0pt}{3ex}\\
经济条件富裕 & 9651 & \begin{tabular}[t]{@{}c@{}} 0.068 \\ (0.003) \end{tabular} & 3403 & \begin{tabular}[t]{@{}c@{}} 0.065 \\ (0.004) \end{tabular} & 1154 & \begin{tabular}[t]{@{}c@{}} 0.049 \\ (0.006) \end{tabular} &     0.003 &     0.019** &     0.017** \rule{0pt}{3ex}\\
\hline  \\[-1.8ex]
%%% This is the note. If it does not have the correct margins, edit text below to fit to table size.
\multicolumn{10}{@{}p{1\textwidth}}

\end{tabular}}
\end{table}
\end{frame}



\begin{frame}
\begin{table}[!htbp]
\centering
\caption{与祖辈同住对青少年认知能力的影响}
\label{cog1}
\resizebox{140pt}{110pt}{
\begin{tabular}{lccc}
\hline
		    & \multicolumn{3}{c}{认知能力}  \\

             & (1)       & (2)        & (3)       \\
\hline
             &           &            &           \\
三代同住         & 0.0627*** & 0.0584***  & 0.0503*** \\
             & (0.0163)  & (0.0161)   & (0.0154)  \\
隔代同住 		 & -0.0629** & -0.0370    & 0.0203    \\
             & (0.0257)  & (0.0254)   & (0.0251)  \\
年龄           & -0.189*** & -0.163***  & -0.118*** \\
             & (0.0100)  & (0.0100)   & (0.0118)  \\
少数民族         & -0.109*** & -0.0805*** & -0.0558   \\
             & (0.0257)  & (0.0255)   & (0.0344)  \\
女性           & 0.00951   & 0.00333    & -0.0146   \\
             & (0.0138)  & (0.0136)   & (0.0136)  \\
九年级           & 0.368***  & 0.321***   & 0.255***  \\
             & (0.0244)  & (0.0243)   & (0.0398)  \\
农村户口           & -0.135*** & -0.0409*** & 0.0216    \\
             & (0.0148)  & (0.0155)   & (0.0189)  \\
上过学前班        & 0.171***  & 0.142***   & 0.108***  \\
             & (0.0179)  & (0.0178)   & (0.0180)  \\
流动儿童         & 0.00648   & 0.00690    & -0.0163   \\
             & (0.0179)  & (0.0178)   & (0.0249)  \\
独生子女         & 0.234***  & 0.162***   & -0.00455  \\
             & (0.0151)  & (0.0155)   & (0.0197)  \\
父母的最高受教育年限   &           & 0.0408***  & 0.0248*** \\
             &           & (0.00264)  & (0.00357) \\
经济条件中等       &           & 0.130***   & 0.0371**  \\
             &           & (0.0180)   & (0.0178)  \\
经济条件富裕       &           & 0.181***   & 0.00849   \\
             &           & (0.0314)   & (0.0328)  \\
Constant     & 2.440***  & 1.542***   & 1.258***  \\
             & (0.140)   & (0.147)    & (0.181)   \\
             &           &            &          \\
学校固定效应  & 否    	& 否     	& 是  \\
R-squared    & 0.094     & 0.115      & 0.243 \\
Observations & 14,208    & 14,208     & 14,208    \\
\hline
\end{tabular}}
\end{table}
\end{frame}

\begin{frame}
	\frametitle{Result:与祖辈同住对青少年认知能力的影响}
表\ref{cog1}汇报了与祖辈同住和青少年认知能力之间的关系。第(1)列的简单线性回归结果说明,三代同住家庭对学生认知能力存在显著的正向影响。在(1)列的基础上,第(2)列进一步控制了父母受教育程度和家庭经济状况,第(3)列增加了学校层面固定效应。实证结果显示,三代同住家庭的学生相较亲代同住家庭学生的学业表现更好,而隔代同住家庭的学生与亲代同住家庭学生在认知能力上无显著差异。基于第(3)列的估计结果本文认为,三代同住家庭会使得学生的认知能力提高0.05个标准差,隔代同住对学生认知能力无显著影响,这与\citet{pong_co-resident_2010}使用我国台湾样本的结论不一致,可能的原因在与样本的不同。
\end{frame}



\begin{frame}
\begin{table}[!htbp]
\centering
\caption{与祖辈同住对青少年非认知能力的影响}
\label{nocog}
%\resizebox{\textwidth}{!}{
\resizebox{270pt}{110pt}{

\begin{tabular}{lccccccccc}
\hline
	    & \multicolumn{3}{c}{情绪稳定性}    & \multicolumn{3}{c}{宜人性}  & \multicolumn{3}{c}{尽责性}      \\
             & (1)        & (2)        & (3)        & (4)        & (5)        & (6)       & (7)        & (8)        & (9)       \\
\hline
三代同住         & 0.0454**   & 0.0421**   & 0.0513**   & 0.0602***  & 0.0557***  & 0.0603*** & 0.0489**   & 0.0497**   & 0.0407*   \\
             & (0.0199)   & (0.0198)   & (0.0226)   & (0.0195)   & (0.0194)   & (0.0188)  & (0.0195)   & (0.0195)   & (0.0222)  \\
隔代同住 & -0.175***  & -0.164***  & -0.127***  & -0.121***  & -0.0970*** & -0.0457   & 0.0107     & 0.00696    & -0.00163  \\
             & (0.0312)   & (0.0312)   & (0.0345)   & (0.0307)   & (0.0305)   & (0.0329)  & (0.0307)   & (0.0308)   & (0.0333)  \\
年龄           & -0.0678*** & -0.0584*** & -0.0412*** & -0.0614*** & -0.0388*** & -0.0136   & -0.0180    & -0.0216*   & -0.0128   \\
             & (0.0122)   & (0.0123)   & (0.0128)   & (0.0120)   & (0.0120)   & (0.0143)  & (0.0120)   & (0.0122)   & (0.0112)  \\
少数民族         & -0.0573*   & -0.0366    & 0.0453     & -0.185***  & -0.153***  & -0.0389   & -0.0106    & -0.0157    & -0.00831  \\
             & (0.0313)   & (0.0313)   & (0.0346)   & (0.0308)   & (0.0306)   & (0.0408)  & (0.0308)   & (0.0309)   & (0.0473)  \\
女性           & -0.0684*** & -0.0723*** & -0.0774*** & 0.155***   & 0.150***   & 0.144***  & 0.241***   & 0.242***   & 0.239***  \\
             & (0.0167)   & (0.0167)   & (0.0180)   & (0.0164)   & (0.0163)   & (0.0155)  & (0.0165)   & (0.0165)   & (0.0178)  \\
九年级           & -0.0865*** & -0.107***  & -0.138***  & 0.0751**   & 0.0316     & -0.00432  & -0.285***  & -0.279***  & -0.289*** \\
             & (0.0297)   & (0.0298)   & (0.0360)   & (0.0292)   & (0.0292)   & (0.0393)  & (0.0293)   & (0.0294)   & (0.0303)  \\
农村户口          & 0.0415**   & 0.0684***  & 0.0484**   & -0.0688*** & 0.00880    & -0.00700  & 0.0885***  & 0.0765***  & 0.0618*** \\
             & (0.0180)   & (0.0190)   & (0.0214)   & (0.0176)   & (0.0186)   & (0.0203)  & (0.0177)   & (0.0188)   & (0.0190)  \\
上过学前班        & 0.0955***  & 0.0797***  & 0.0671***  & 0.148***   & 0.118***   & 0.103***  & -0.0177    & -0.0131    & -0.00586  \\
             & (0.0218)   & (0.0218)   & (0.0240)   & (0.0214)   & (0.0213)   & (0.0208)  & (0.0214)   & (0.0215)   & (0.0219)  \\
流动儿童         & -0.0153    & -0.0239    & -0.0234    & 0.0204     & 0.0148     & 0.0287    & -0.0812*** & -0.0803*** & -0.0138   \\
             & (0.0218)   & (0.0218)   & (0.0233)   & (0.0214)   & (0.0213)   & (0.0240)  & (0.0215)   & (0.0215)   & (0.0221)  \\
独生子女         & 0.0436**   & 0.0187     & 0.00372    & 0.139***   & 0.0790***  & -0.00980  & 0.00581    & 0.0154     & -0.0156   \\
             & (0.0184)   & (0.0190)   & (0.0224)   & (0.0181)   & (0.0186)   & (0.0206)  & (0.0181)   & (0.0188)   & (0.0219)  \\
父母的最高受教育年限   &            & 0.00445    & 0.00821**  &            & 0.0280***  & 0.0191*** &            & -0.00435   & -0.00324  \\
             &            & (0.00325)  & (0.00392)  &            & (0.00317)  & (0.00374) &            & (0.00320)  & (0.00387) \\
经济条件中等       &            & 0.189***   & 0.160***   &            & 0.194***   & 0.130***  &            & -0.0330    & -0.0564** \\
             &            & (0.0222)   & (0.0263)   &            & (0.0217)   & (0.0206)  &            & (0.0219)   & (0.0231)  \\
经济条件富裕       &            & 0.220***   & 0.159***   &            & 0.350***   & 0.217***  &            & -0.0470    & -0.0843** \\
             &            & (0.0386)   & (0.0432)   &            & (0.0377)   & (0.0384)  &            & (0.0380)   & (0.0387)  \\
Constant     & 0.957***   & 0.637***   & 0.405**    & 0.659***   & -0.108     & -0.256    & 0.266      & 0.386**    & 0.279     \\
             & (0.171)    & (0.181)    & (0.191)    & (0.168)    & (0.177)    & (0.207)   & (0.168)    & (0.178)    & (0.174)   \\
学校固定效应  & 否    	& 否     	& 是 			& 否    	 	& 否     	 & 是    		& 否    		& 否     	& 是  \\
R-squared    & 0.023      & 0.029      & 0.061      & 0.033      & 0.048      & 0.109     & 0.043      & 0.043      & 0.072    \\
Observations & 14,208     & 14,208     & 14,208     & 14,208     & 14,208     & 14,208    & 14,208     & 14,208     & 14,208    \\
\hline
\end{tabular}} %注意这里还有}
\end{table}	
\end{frame}

\begin{frame}
	\frametitle{与祖辈同住对青少年非认知能力的影响}
表\ref{nocog}汇报了与祖辈同住和青少年非认知能力之间的关系。可以发现在控制了学校层面固定效应后,三代同住对青少年各维度的非认知能力均有显著正向影响,隔代同住对青少年宜人性和尽责性无显著影响,但对其情绪稳定性有显著负向影响。
\end{frame}


\begin{frame}
	\frametitle{结论}
可以看出三代同住对于青少年认知能力有显著的促进作用,隔代同住对于青少年认知能力没有显著影响
\par 但从非认知能力来看,三代同住的家庭仍旧显示出对于青少年的正向影响,但是隔代同住对于青少年的情绪稳定性和宜人性有显著的负面影响。
\end{frame}



\begin{frame}
	\frametitle{机制思考}
	\begin{itemize}
		\item 家庭经济条件:祖辈经济资源+父辈的经济资源
		\item 父母社会化:祖辈承担抚养的职能,父辈有更多时间关注孙辈的学习
		\item 家庭不稳定:父母离异等
	\end{itemize}
	从父母社会化出发,比较三类家庭中父母对孩子学习和心理健康的关注程度,
\end{frame}




\begin{frame}
	\frametitle{局限}
	\begin{enumerate}
		\item 数据的限制无法控制祖辈的相关信息(遗漏祖辈相关信息)
		\item 机制部分还比较迷惑,希望各位老师提出建议。
	\end{enumerate}
\end{frame}


\begin{frame}[allowframebreaks] %允许分页面使用
   \frametitle{参考文献}
\tiny
\bibliographystyle{apalike}
\bibliography{references}
\end{frame}
%参考文献的案例 \citet{Krueger1999Experimental} \citep{Krueger1999Experimental}  注意两类的不同


\begin{frame}
	\begin{center}
 		\huge {欢迎大家批评!}\\
		\vspace{1cm}
		\huge {Q\&A}
	\end{center}
\end{frame}
\end{document}



