%*============================================================*
%**Goal		:    文献分享:Long-Term Effects Of The 1959-1961 China Famine: Mainland China and Hong Kong
%**Author	:  	 ZhangYi zhangyiceee@163.com 15592606739
%**Created	:  	 20200513
%**Last Modified: 2020
%*============================================================*



\documentclass{beamer}
\usepackage[UTF8,noindent]{ctexcap}
\graphicspath{{figures/}}


\usetheme{Madrid}
%Information to be included in the title page:
\title[文献分享:ZY]{Long-Term Effects Of The 1959-1961 China Famine: Mainland China and Hong Kong}
\author[Famine\_workshop]{Douglas Almond、Lena Edlund、Hongbin Li \&Junsen Zhang}
\date{\today}

\begin{document}
\frame{\titlepage}
%开始你的表演


\begin{frame}
	\frametitle{Abstract}
This paper estimates the effects of maternal malnutrition exploiting the 1959-1961 Chinese famine as a natural experiment. In the 1\% sample of the 2000 Chinese Census, we find that fetal exposure to acute maternal malnutrition had compromised a range of socioeconomic outcomes, including: literacy, labor market status, wealth and marriage market outcomes. Women married spouses with less education and later, as did men, if at all. In addition, maternal malnutrition reduced the sex ratio (males to females) in two generations -- those prenatally exposed and their children -- presumably through heightened male mortality. This tendency toward female offspring is interpretable in light of the Trivers-Willard (1973) hypothesis, according to which parents in poor condition should skew the offspring sex ratio toward daughters. Hong Kong natality micro data from 1984-2004 further confirm this pattern of female offspring among mainland-born residents exposed to malnutrition in utero.
\end{frame}


\end{document}




