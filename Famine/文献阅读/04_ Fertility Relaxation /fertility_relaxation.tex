%*============================================================*
%**Goal		:    文献分享:消失的女性与茶叶的价格
%**Author	:  	 ZhangYi zhangyiceee@163.com 15592606739
%**Created	:  	 20200323
%**Last Modified: 2020
%*============================================================*



\documentclass{beamer}
\usepackage[UTF8,noindent]{ctexcap}
\usepackage{natbib}
\usepackage{hyperref}


\graphicspath{{figures/}}




\usetheme{Madrid}
%\usecolortheme{crane} %黄色
%\usecolortheme{wolverine} %黄蓝
%\usecolortheme{lily} %蓝白
%\usecolortheme{beetle} %灰色
%Information to be included in the title page:
\title[文献分享:ZY]{The Impact of Fertility Relaxation on Female Labor Market Outcomes}
\author[Sumit et al.]{Sumit Agarwal, Keyang Li , Yu Qin \& Jing Wu}
\date{\today}

\begin{document}

\frame{\titlepage}
%开始你的表演



%第1页幻灯片
%==========================================
\begin{frame}
\frametitle{Abstract}
We explore a 2013 policy shock that relaxed the One-Child Policy in China: couples could have two children under certain circumstances. We show that after the policy shock the salary of female new hires is reduced by 1.2\% relative to the salary of male new hires, equivalent to a 22\% increase in the gender wage gap in the data. In addition, employers hire 4.4\% fewer female employees, and female employees are 4.3\% less likely to leave their current jobs. This leads to approximately 1,950 fewer female employees employed and 1,059 leaving their jobs every month in the sample city.
\end{frame}

\begin{frame}
	\frametitle{}
	
\end{frame}


\begin{frame}
	\frametitle{}
	
\end{frame}

\begin{frame}
	\frametitle{}
	
\end{frame}


\begin{frame}
	\frametitle{}
	
\end{frame}


\begin{frame}
	\frametitle{}
	
\end{frame}


\begin{frame}
	\frametitle{}
	
\end{frame}

\end{document}



